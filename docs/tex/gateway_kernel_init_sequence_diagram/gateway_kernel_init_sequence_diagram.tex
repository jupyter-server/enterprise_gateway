\documentclass[convert]{standalone}
\usepackage{fontspec}
\usepackage{pgf-umlsd}
\usepackage[bottom]{footmisc}

\setmainfont{Consolas}

\begin{document}
\begin{sequencediagram}
    \newthread{browser}{:browser}
    \newinst[4]{server}{:jupyter\_server}
    \newinst[3]{gateway}{:enterprise\_gateway}
    \newinst[4]{kernel}{:kernel}
    
    \begin{sdblock}{fetch kernelspecs}{}
        \begin{call}{browser}{https GET api/kernelspecs}{server}{api/kernelspecs response}
            \begin{call}{server}{https GET api/kernelspecs}{gateway}{api/kernelspecs response}
            \postlevel
            \end{call}
        \end{call}
    \end{sdblock}
    
    \begin{sdblock}{kernel init}{}
        \begin{call}{browser}{https POST api/sessions}{server}{api/sessions response}
            \begin{call}{server}{https POST api/kernels}{gateway}{api/kernels response}
                \begin{call}{gateway}{RemoteProcessProxy.launch\_process}{kernel}{kernel connection info\footnote{see https://github.com/jupyter/enterprise\_gateway/pull/877}}
                    \postlevel
                \end{call}
                \postlevel
                \begin{call}{gateway}{TCP socket requests}{kernel}{TCP socket handshakes}
                    \postlevel
                \end{call}
            \end{call}
        \end{call}
        \postlevel
        \begin{call}{browser}{ws GET api/kernels}{server}{websocket upgrade response}
            \begin{call}{server}{ws GET api/kernels}{gateway}{websocket upgrade response}
            \postlevel
            \end{call}
        \end{call}
    \end{sdblock}
\end{sequencediagram}
\end{document}